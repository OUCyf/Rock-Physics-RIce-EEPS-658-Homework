
%%%%%%%%%%%%%%%%%%%%%%%%%%%%%%%%%%%%%%%%%%%%%%%%%%%
%                  1. Package
%%%%%%%%%%%%%%%%%%%%%%%%%%%%%%%%%%%%%%%%%%%%%%%%%%%
% \usepackage{ctex} % chinese
\usepackage{geometry}
\usepackage{fancyhdr} % header and footer
\usepackage{graphicx} % insert fig
\usepackage{pdfpages} % insert pdf
\usepackage{multicol} % muti-column
\usepackage{multirow} 
\usepackage{setspace} % line spacing
\usepackage{makeidx} % index
\usepackage{tikz} % plot env
\usepackage{indentfirst} % \chapter、\section indentation for the first line
\usepackage{caption,subcaption}  % title and subtitle
\usepackage{amsmath,amsthm,amssymb,amsfonts,mathrsfs} % math
\usepackage{siunitx}  % SI unit
\usepackage{pifont} % special symbol
\usepackage{hyperref} % link
\usepackage{cleveref}
\usepackage{booktabs} % three line table
\usepackage{tabularx} % three line table
\usepackage{longtable}
\usepackage[ruled,linesnumbered]{algorithm2e}  % algorithm
\usepackage[shortlabels]{enumitem} % list env
\usepackage{xcolor} % font color
\usepackage{graphbox} % figure position
\usepackage{minted} % code style
% \usepackage{listings} % code style (not recommended)
\usepackage{environ}
\usepackage{mdframed}

\usepackage[round]{natbib} % bib - authoryear
\bibliographystyle{plainnat} %  plainnat

% \usepackage{titlesec}    % Remove the text 'Chapter' from each chapter.
% \titleformat{\chapter}{\normalfont\huge\bf}{\thechapter}{20pt}{\huge\bf}



%%%%%%%%%%%%%%%%%%%%%%%%%%%%%%%%%%%%%%%%%%%%%%%%%%%
%                  2. Page Setting
%%%%%%%%%%%%%%%%%%%%%%%%%%%%%%%%%%%%%%%%%%%%%%%%%%%
% margin
\geometry{left=2.5cm,right=2.5cm,top=2.5cm,bottom=2.5cm}

% footer and header
\pagestyle{fancy}
\fancyfoot[C]{\thepage}

% spaing
\linespread{1.2}  % line spacing || 1.2 times font size
\setlength{\parindent}{2em} % indentation for the first line
\setlength{\parskip}{1ex} % spacing for each paragraph

% link color
\hypersetup{
    colorlinks = true,
    citecolor = blue, % References
    linkcolor = black,
    filecolor = magenta,
    urlcolor = {blue!40!black},
}





%%%%%%%%%%%%%%%%%%%%%%%%%%%%%%%%%%%%%%%%%%%%%%%%%%%
%                  3. Plot Env
%%%%%%%%%%%%%%%%%%%%%%%%%%%%%%%%%%%%%%%%%%%%%%%%%%%
\usetikzlibrary{trees,positioning,fit,calc}
\tikzset{block/.style = {draw, fill=blue!20, rectangle,
                         minimum height=3em, minimum width=4em},
        input/.style = {coordinate},
        output/.style = {coordinate}
}






%%%%%%%%%%%%%%%%%%%%%%%%%%%%%%%%%%%%%%%%%%%%%%%%%%%
%                  4. Question Env
%%%%%%%%%%%%%%%%%%%%%%%%%%%%%%%%%%%%%%%%%%%%%%%%%%%
\newenvironment{problem}[2][Problem]
    {\begin{mdframed}[backgroundcolor=gray!10] \textbf{#1 #2} \\}
    {\end{mdframed}}

\newenvironment{solution}{\textbf{Solution}}






%%%%%%%%%%%%%%%%%%%%%%%%%%%%%%%%%%%%%%%%%%%%%%%%%%%
%     5. Code Env || use 'minted' and 'pygments'
%%%%%%%%%%%%%%%%%%%%%%%%%%%%%%%%%%%%%%%%%%%%%%%%%%%
\usemintedstyle{vs}  % minted code style refer to https://pygments.org/styles/

% mathescape 允许代码段中插入公式
% linenos 代码行数
% numbersep 设置行号和行首的间距
% breaklines 自动折断换行
% bgcolor 背景色
% frame 设置代码周围的线
% framerule 设置框线的宽度
% framesep 设置框线与内容的间距

\newminted{python}{
    mathescape,
    linenos=true, 
    breaklines=true,
    bgcolor=gray!10,
    numbersep=5pt,
    frame=lines,
    framesep=2mm,
}

\newmintedfile{python}{
    mathescape,
    linenos=true, 
    breaklines=true,
    bgcolor=gray!10,
    numbersep=5pt,
    frame=lines,
    framesep=2mm,
}

\newminted{julia}{
    mathescape,
    linenos=true, 
    breaklines=true,
    bgcolor=gray!10,
    numbersep=5pt,
    frame=lines,
    framesep=2mm,
}

\newmintedfile{julia}{
    mathescape,
    linenos=true, 
    breaklines=true,
    bgcolor=gray!10,
    numbersep=5pt,
    frame=lines,
    framesep=2mm,
}

%%%%%%%%%%%%%%%%%%%%%%%%%%%%%%%%%%%
% listing package is not recommended now, showing below
%%%%%%%%%%%%%%%%%%%%%%%%%%%%%%%%%%%
% \lstset{
%     basicstyle          =   {\small\ttfamily},
%     keywordstyle        =   \bfseries,
%     commentstyle        =   \rmfamily\itshape,
%     stringstyle         =   \ttfamily,
%     flexiblecolumns,
%     numbers             =   none,
%     showspaces          =   false,
%     numberstyle         =   \small\ttfamily,
%     showstringspaces    =   false,
%     captionpos          =   t,
%     frame               =   tb,
% }

% \lstdefinestyle{Python}{
%     language        =   Python,
%     keywordstyle    =   \color{blue},
%     keywordstyle    =   [2] \color{teal},
%     stringstyle     =   \color{magenta},
%     commentstyle    =   \color{gray}\ttfamily,
%     breaklines      =   true,
%     columns         =   fixed,
%     basewidth       =   0.5em,
% }

% \lstdefinelanguage{Julia}%
%   {morekeywords={abstract,break,case,catch,const,continue,do,else,elseif,%
%       end,export,false,for,function,immutable,import,importall,if,in,%
%       macro,module,otherwise,quote,return,switch,true,try,type,typealias,%
%       using,while},%
%    sensitive=true,%
%    alsoother={$},%
%    morecomment=[l]\#,%
%    morecomment=[n]{\#=}{=\#},%
%    morestring=[s]{"}{"},%
%    morestring=[m]{'}{'},%
% }[keywords,comments,strings]%

% \lstdefinestyle{Julia}{
%     language        =   Julia,
%     keywordstyle    =   \color{blue},
%     keywordstyle    =   [2] \color{teal},
%     stringstyle     =   \color{magenta},
%     commentstyle    =   \color{gray}\ttfamily,
%     breaklines      =   true,
%     columns         =   fixed,
%     basewidth       =   0.5em,
% }





%%%%%%%%%%%%%%%%%%%%%%%%%%%%%%%%%%%%%%%%%%%%%%%%%%%
%                  6. Fig Env
%%%%%%%%%%%%%%%%%%%%%%%%%%%%%%%%%%%%%%%%%%%%%%%%%%%
\graphicspath{figures/}
\captionsetup[figure]{
  labelfont={bf},
  labelformat={default},
  name={Figure.},
  labelsep=colon
}





%%%%%%%%%%%%%%%%%%%%%%%%%%%%%%%%%%%%%%%%%%%%%%%%%%%
%                 7. Footnote Env
%%%%%%%%%%%%%%%%%%%%%%%%%%%%%%%%%%%%%%%%%%%%%%%%%%%
% \interfootnotelinepenalty=10000 % 禁止脚注显示在第二页
% % redefine footnote as circled number: ①②③④⑤⑥⑦⑧⑨
% \renewcommand{\thefootnote}{\ding{\numexpr171+\value{footnote}}}
